\section{What is the double beta decay?}
\begin{frame}{Beta Decay}
	\begin{columns}
		\begin{column}{.5\textwidth}
			A small reminder:\\
			\begin{centering}
				\begin{tabular}{c S[table-format=3.0]}
					Particle                       & {Mass/\si{\kilo\electronvolt}} \\
					\toprule
					$\symup{n}$                    & 939565                          \\
					$\symup{p}$                    & 938272                          \\
					$\symup{n-p}$                  & 1293                            \\
					$\symup{e^-}$                  & 511                             \\
					$\symup{\symup{\anti\nu_{e}}}$ & 0                               \\
				\end{tabular}\\
			\end{centering}
			\vspace{1em}
			The $\beta^-$--Decay is energetically possible:
			\begin{equation*}
				\symup{n^0}\to\symup{p^+}+\symup{e^-}+\symup{\anti\nu_{e}}
			\end{equation*}
		\end{column}
		\begin{column}{.5\textwidth}
			\begin{tikzpicture}
				\begin{feynman}
					\vertex (d1) {$\symup{d}$};
					\vertex[right=5 of d1] (d2) {$\symup{d}$};
					\vertex[below=2em of d1] (u1) {$\symup{u}$};
					\vertex[right=5 of u1] (u2) {$\symup{u}$};
					\vertex[below=2em of u1] (d3) {$\symup{d}$};
					\vertex[right=5 of d3] (u3) {$\symup{u}$};
					\vertex[right=2.5 of d3] (v1);
					\vertex[below=1 of v1] (v2);
					\vertex[above right=0.2 and 2.25 of v2] (nu) {$\symup{\anti\nu_{e}}$};
					\vertex[below right=0.2 and 2.25 of v2] (e) {$\symup{e^-}$};
					\diagram* { {[edges=fermion]
							(d1) -- (d2),  (u1) -- (u2),
							(d3) -- (v1) -- (u3), (nu) -- (v2) -- (e)},
					(v1) -- [boson, edge label'=$\symup{W^-}$] (v2)
					};
					\draw [decoration={brace}, decorate] (d3.south west) -- (d1.north west) node [pos=0.5, left] {$\symup{n}$};
					\draw [decoration={brace}, decorate] (d2.north east) --  (u3.south east) node [pos=0.5, right] {$\symup{p}$};
				\end{feynman}
			\end{tikzpicture}
		\end{column}
	\end{columns}
	\center\onslide<2->{\alert{ Why are neutrons in nuclei stable?}}
\end{frame}

\begin{frame}
	\frametitle{Energy levels inside the nucleus}
	\centering
	\vspace{-1em}
	\begin{figure}
		\centering
		\only<1>{\input{build/const_a_all.pgf}}
		\only<2>{\input{build/const_a_zoom.pgf}}
		\only<3>{\input{build/const_a_betabeta.pgf}}
		\caption*{Isobar energy levels for $A=76$\footnote[1]{Data from: \fullcite{Huang2021}}}%
		\label{fig:constA}
	\end{figure}
\end{frame}

\begin{frame}
	\frametitle{Energy levels inside the nucleus}
	The parabolic shape can be explained in the liquid-drop-model (Bethe-Weizsäcker-formula):
	\begin{equation*}
		E=a_VA- a_OA^\frac{2}{3}- a_C\frac{Z^2}{A^\frac{1}{3}}
		-a_{\text{A}}\frac{\left(A-2Z\right)^2}{A}
		+\begin{cases}
			+a_{\text{P}}A^{-\frac{1}{2}} & Z,N \text{ even } (A \text{ even}) \\
			0                             & A \text{ odd}                      \\
			-a_{\text{P}}A^{-\frac{1}{2}} & Z,N \text{ odd } (A \text{ even})
		\end{cases}
	\end{equation*}
	\begin{itemize}
		\item<2-> For even $A$, there are actually two parabolas
		%\item<3-> Double-$\beta$-decay is very rare. Other decays must be forbidden to observe it.
		%\begin{equation*}
		%    T_{\sfrac{1}{2}}^{\beta\beta}>\SI{1e18}{\year}>\SI{13.8e9}{\year}=T_{\symup{universe}}
		%\end{equation*}
		\item<3-> Hence we expect $\beta\beta$ decay only in even-$N$ even-$Z$ nuclei\\
		$\symup{{}^{48}_{20}Ca,
				{}^{76}_{32}Ge,
				{}^{78}_{36}Kr,
				{}^{82}_{34}Se,
				{}^{96}_{40}Zr,
				{}^{100}_{42}Mo,
				{}^{116}_{48}Cd,
				{}^{128}_{52}Te,
				{}^{130}_{52}Te,
				{}^{124}_{54}Xe,
				{}^{136}_{54}Xe,
				{}^{130}_{56}Ba,
				{}^{150}_{60}Nd,
				{}^{238}_{92}U}$
	\end{itemize}
\end{frame}

\begin{frame}[t]
	\frametitle{Relevance for particle physics}
	\begin{columns}
		\begin{column}{.55\textwidth}
			\begin{itemize}
				\item $(A, Z) \to (A, Z+2) + 2\symup{e^-}+2\anti\nu_{\symup{e}}$ is allowed in the SM
				      %\begin{itemize}
				      %	\item Process can be desribed with an virtual (energeticly forbidden) intermediate state $(A, Z+1)+\symup{e^-}+\anti\nu_{\symup{e}}$
				      %\end{itemize}
				\item $(A, Z) \to (A, Z+2) + 2\symup{e^-}$ is forbidden in the SM
				      \begin{itemize}
					      \item Violates lepton number conservation\\ \alert{$\rightarrow$highly interesting}
					            \item<2-> Solution: \emph{Majorana}-Neutrinos
					            \begin{align*}
						            \label{eq:}
						            \symcal{L} = \frac{1}{2} \anti\psi\ \l(\slashed{p}-m\r)\psi\quad\text{and}\quad \psi = \gamma_0 C \psi^*
					            \end{align*}
							\item<3-> Can be used to gain knowledge about the absolute scale of $\nu$-masses (impossible via $\nu$-oscillation)
									\begin{align*}
										\l(T^{0\nu}_{\sfrac{1}{2}}\r)^{-1} = G_{0\nu} \abs{M_{0\nu}}^2\abs{m_{\beta\beta}}^2 \quad \text{with} \quad m_{\beta\beta} = \sum_{i=1}^{3}{m_i U_{ei}^2}
									\end{align*}
				      \end{itemize}
			\end{itemize}
		\end{column}
		\begin{column}{.4\textwidth}
			\only<1>{
				\begin{tikzpicture}
					\begin{feynman}
						\vertex (d1) {$\symup{d}$};
						\vertex[right=5 of d1] (d2) {$\symup{d}$};
						\vertex[below=2em of d1] (u1) {$\symup{u}$};
						\vertex[right=5 of u1] (u2) {$\symup{u}$};
						\vertex[below=2em of u1] (d3) {$\symup{d}$};
						\vertex[right=5 of d3] (u3) {$\symup{u}$};
						\vertex[right=2.5 of d3] (v1);
						\vertex[below=0.7 of v1] (v2);
						\vertex[below right=0.15 and 2.25 of v2] (nu) {$\symup{\anti\nu_{e}}$};
						\vertex[right=2.25 of v2] (e) {$\symup{e^-}$};
						\diagram* { {[edges=fermion]
								(d1) -- (d2),  (u1) -- (u2),
								(d3) -- (v1) -- (u3), (nu) -- (v2) -- (e)},
						(v1) -- [boson, edge label'=$\symup{W^-}$] (v2)
						};
						\draw [decoration={brace}, decorate] (d3.south west) -- (d1.north west) node [pos=0.5, left] {$\symup{n}$};
						\draw [decoration={brace}, decorate] (d2.north east) --  (u3.south east) node [pos=0.5, right] {$\symup{p}$};

						\vertex[below=7em of d3] (d21) {$\symup{d}$};
						\vertex[right=5 of d21] (d22) {$\symup{d}$};
						\vertex[below=2em of d21] (u21) {$\symup{u}$};
						\vertex[right=5 of u21] (u22) {$\symup{u}$};
						\vertex[below=2em of u21] (d23) {$\symup{d}$};
						\vertex[right=5 of d23] (u23) {$\symup{u}$};
						\vertex[right=2.5 of d21] (v21);
						\vertex[above=0.7 of v21] (v22);
						\vertex[above right=0.15 and 2.25 of v22] (nu2) {$\symup{\anti\nu_{e}}$};
						\vertex[right=2.25 of v22] (e2) {$\symup{e^-}$};
						\diagram* { {[edges=fermion]
								(d21) -- (v21) -- (d22),  (u21) -- (u22),
								(d23) -- (u23), (nu2) -- (v22) -- (e2)},
						(v21) -- [boson, edge label=$\symup{W^-}$] (v22)
						};
						\draw [decoration={brace}, decorate] (d23.south west) -- (d21.north west) node [pos=0.5, left] {$\symup{n}$};
						\draw [decoration={brace}, decorate] (d22.north east) --  (u23.south east) node [pos=0.5, right] {$\symup{p}$};
					\end{feynman}
				\end{tikzpicture}
			}
			\only<2->{
				\begin{tikzpicture}
					\begin{feynman}
						\vertex (d1) {$\symup{d}$};
						\vertex[right=5 of d1] (d2) {$\symup{d}$};
						\vertex[below=2em of d1] (u1) {$\symup{u}$};
						\vertex[right=5 of u1] (u2) {$\symup{u}$};
						\vertex[below=2em of u1] (d3) {$\symup{d}$};
						\vertex[right=5 of d3] (u3) {$\symup{u}$};
						\vertex[right=2.5 of d3] (v1);
						\vertex[below=0.7 of v1] (v2);
						\vertex[right=2.25 of v2] (e) {$\symup{e^-}$};
						\draw [decoration={brace}, decorate] (d3.south west) -- (d1.north west) node [pos=0.5, left] {$\symup{n}$};
						\draw [decoration={brace}, decorate] (d2.north east) --  (u3.south east) node [pos=0.5, right] {$\symup{p}$};

						\vertex[below=7em of d3] (d21) {$\symup{d}$};
						\vertex[right=5 of d21] (d22) {$\symup{d}$};
						\vertex[below=2em of d21] (u21) {$\symup{u}$};
						\vertex[right=5 of u21] (u22) {$\symup{u}$};
						\vertex[below=2em of u21] (d23) {$\symup{d}$};
						\vertex[right=5 of d23] (u23) {$\symup{u}$};
						\vertex[right=2.5 of d21] (v21);
						\vertex[above=0.7 of v21] (v22);
						\vertex[right=2.25 of v22] (e2) {$\symup{e^-}$};
						\diagram* { {[edges=fermion]
								(d1) -- (d2),  (u1) -- (u2),
								(d3) -- (v1) -- (u3), (v2) -- (e),
								(d21) -- (v21) -- (d22),  (u21) -- (u22),
								(d23) -- (u23), (v22) -- (e2)},
						(v22) -- [majorana, edge label'=$\symup{\nu_e}$] (v2),
						(v21) -- [boson, edge label=$\symup{W^-}$] (v22),
						(v1) -- [boson, edge label'=$\symup{W^-}$] (v2),
						};
						\draw [decoration={brace}, decorate] (d23.south west) -- (d21.north west) node [pos=0.5, left] {$\symup{n}$};
						\draw [decoration={brace}, decorate] (d22.north east) --  (u23.south east) node [pos=0.5, right] {$\symup{p}$};
					\end{feynman}
				\end{tikzpicture}
			}
		\end{column}
	\end{columns}
\end{frame}
